% Preamble
\documentclass[11pt]{article}
	% Necessary packages
	\usepackage[utf8]{inputenc}
	\usepackage[english]{babel}
	\usepackage{graphicx}
	\usepackage{fancyhdr}
	\usepackage{xcolor}
	\usepackage{lipsum}
	\usepackage{amsmath}
	\usepackage{float}
	\usepackage{natbib}
	\usepackage{url}
	\usepackage{hyperref}
	\usepackage{mathptmx}
	\usepackage{amsmath,amssymb,amsfonts}
	% Page setup
	\usepackage{geometry}
	\geometry{
		a4paper,
		left=2cm,
		right=2cm,
		top=2.5cm,
		bottom=2.5cm,
	}
	% Header and footer
	\pagestyle{fancy}
	\fancyhf{}
	\lhead{Master’s degree in economics}
	\chead{Lecture Notes}
	\rhead{Jhon R. Ordoñez}
	\rfoot{Page \thepage}
% Body
\begin{document}
	% Title page
	\begin{titlepage}
		\vspace*{5cm}
		% Title
		{\Large
			\begin{center}
				{\Huge \textbf{DIFFERENTIAL EQUATIONS}}
				\vspace{0.5cm}\\
				% Name
				Jhon Roly Ordoñez Leon\\
				(\textcolor{blue}{f1352289@pucp.edu.pe})
				\vspace{0.5cm}\\
				% Topic
				\textbf{First year master in economics}
			\end{center}
		}
		% Date
		\date{\today}
		% Abstract
		\begin{abstract}
			\lipsum[1].\\
			\textbf{Key words:} \hspace{1mm} \textit{word 1, word2.}
		\end{abstract}
		% Table of contents
		\textcolor{red}{\tableofcontents}
	\end{titlepage}

	% Section 1
	\section{Introduction to Differential Equations}
		% Subsection 1
		\subsection{Definition of a Differential Equation}
			A differential equation is a mathematical equation
			that relates one or more functions and their derivatives.
			In other words, it describes the relationship between a function
			and its rate of change. The general form of a differential equation is:
				\begin{equation}
					F(x,y,y',y'', \dots, y^{(n)})
				\end{equation}
			Here:
			\begin{itemize}
				\item $x$ is the independent variable.
				\item $y$ is the dependent variable.
				\item $y'$ is the first derivative of $y$ with respect to $x$.
				\item $y''$ is the second derivative, and so on, up to the $n$-th derivative.
 			\end{itemize}
			The order of a differential equation is determined by the highest derivative present.
			Differential equations are fundamental in modeling physical phenomena and natural 
			processes where rates of change play a crucial role. They are widely used in physics, 
			engineering, \textcolor{blue}{economics}, biology, and many other scientific disciplines.
			Differential equations can be classified into several types:
				\begin{enumerate}
					\item \textbf{Ordinary Differential Equations (ODEs):} Involving a single independent variable and its derivatives.
					\item \textbf{Partial Differential Equations (PDEs):} Involving multiple independent variables and their partial derivatives.
					\item \textbf{Linear Differential Equations:} Where the dependent variable and its derivatives appear as linear terms.
					\item \textbf{Nonlinear Differential Equations:} Where the dependent variable and its derivatives appear in a nonlinear manner.
				\end{enumerate}
			Solving a differential equation typically involves finding a function or a set of functions 
			that satisfy the equation. Different methods, such as separation of variables, substitution,
			and using special functions, are employed based on the type and
			complexity of the differential equation.
		% Subsection 2
		\subsection{Types of Differential Equations}
			% Subsubsection 2.1
			\subsubsection{Ordinary Differential Equations (ODEs)}
				Ordinary Differential Equations (ODEs) can be classified into various types based on their
				properties, structure, and characteristics. Here are some common types of ODEs:
			% Subsubsection 2.2
			\subsubsection{Partial Differential Equations (PDEs)}
				\lipsum[1]
	% Section 2
	\section{First-Order Ordinary Differential Equations (ODEs)}
		\lipsum[1]
		% Subsection 1
		\subsection{Separable Equations}
			\lipsum[1]
		% Subsection 2
		\subsection{Linear Equations}
			\lipsum[1]
		% Subsection 3
		\subsection{Exact Equations}
			\lipsum[1]
		% Subsection 4
		\subsection{Integrating Factors}
			\lipsum[1]
		% Subsection 5
		\subsection{Modeling with First-Order ODEs}
			\lipsum[1]
	% Section 3
	\section{Second-Order Ordinary Differential Equations (ODEs)}
		\lipsum[1]
		% Subsection 1
		\subsection{Homogeneous Linear ODEs with Constant Coefficients}
			\lipsum[1]
		% Subsection 2
		\subsection{Non-homogeneous Linear ODEs with Constant Coefficients}
			\lipsum[1]
		% Subsection 3
		\subsection{Undetermined Coefficients}
			\lipsum[1]
		% Subsection 4
		\subsection{Variation of Parameters}
			\lipsum[1]
		% Subsection 5
		\subsection{Modeling with Second-Order ODEs}
			\lipsum[1]
	% Section 4
	\section{Systems of Ordinary Differential Equations}
		\lipsum[1]
		% Subsection 1
		\subsection{Introduction to Systems}
			\lipsum[1]
		% Subsection 2
		\subsection{Linear Systems}
			\lipsum[1]
			% Subsubsection 2.1
			\subsubsection{Homogeneous Systems}
				\lipsum[1]
			% Subsubsection 2.2
			\subsubsection{Non-homogeneous Systems}
				\lipsum[1]
		% Subsection 3
		\subsection{Phase Plane Analysis}
			\lipsum[1]
		% Subsection 4
		\subsection{Modeling with Systems of ODEs}
			\lipsum[1]
	% Section 5
	\section{Laplace Transform}
		\lipsum[1]
		%A. Definition and Properties
		%B. Solving ODEs with Laplace Transform
		%C. Inverse Laplace Transform
		%D. Application to Initial Value Problems
	% Section 6
	\section{Fourier Series and Partial Differential Equations (PDEs)}
		\lipsum[1]
		%A. Fourier Series
		%B. Heat Equation
		%C. Wave Equation
		%D. Laplace's Equation
		%E. Separation of Variables
	% Section 7
	\section{Numerical Methods for Solving Differential Equations}
		\lipsum[1]
		%A. Euler's Method
		%B. Runge-Kutta Methods
		%C. Finite Difference Methods
		%D. Introduction to Finite Element Methods
	% Section 8
	\section{Stability and Bifurcation}
		\lipsum[1]
		%A. Stability of Equilibria
		%B. Bifurcation Analysis
		%C. Applications to Dynamical Systems
	% Section 9
	\section{Introduction to Nonlinear Differential Equations}
		\lipsum[1]
		%A. Nonlinear ODEs
		%B. Nonlinear Systems
		%C. Chaos and Nonlinear Dynamics
	% Section 10
	\section{Applications of Differential Equations}
		\lipsum[1]
		%A. Physics and Engineering Applications
		%B. Biological and Population Dynamics
		%C. Economics and Finance
	% Section 11
	\section{Conclusion}
		\lipsum[1]	
		%A. Recap of Key Concepts
		%B. Further Areas of Study
	% References and bibliography
	\bibliography{main.bib}
		\bibliographystyle{apalike2}
			\nocite{zill-2012}
\end{document}




