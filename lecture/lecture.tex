% Preamble
\documentclass[aspectratio=169]{beamer}
    % Import packages
    \usepackage[utf8]{inputenc}
    \usepackage{multimedia}
    \usepackage{hyperref}
    \usepackage{xcolor}
    \usepackage{graphicx}
    \usepackage{lipsum}
    \usepackage[english]{babel}
    % Beamer theme customization
    \usetheme{metropolis}
    \setbeamertemplate{frame numbering}[fraction]
    \definecolor{CambridgeRed}{RGB}{24,111,101} % Adjust color to match Cambridge style 163,21,45
    \setbeamercolor{progress bar}{fg=CambridgeRed,bg=white}
    \setbeamercolor{section in toc}{fg=black} % Adjust section color
    \setbeamercolor{alerted text}{fg=CambridgeRed} % Adjust alert text color
    \setbeamercolor{block title}{bg=CambridgeRed,fg=white} % Adjust block title colors
    % Table of contents customization
    \setbeamertemplate{section in toc}[sections numbered]
    \setbeamertemplate{subsection in toc}[subsections numbered]
    \setbeamercolor{section in toc}{fg=CambridgeRed}
    \setbeamercolor{subsection in toc}{fg=CambridgeRed}
    \setbeamertemplate{section in toc shaded}[default][50]
    \setbeamertemplate{subsection in toc shaded}[default][50]
    % Title page
    \title{Differential Equations}
    \subtitle{Master's degree in economics}
    \author[JR]{Jhon Roly Ordoñez Leon}
    \date{\today}
    \institute{
        National University of Peru\\
        Faculty of Economic, Administrative and Accounting Sciences\\
        Professional School of Economics
    }
    \titlegraphic{ \hfill \includegraphics[scale=0.3]{img/logo_julia.png}}
% Body
\begin{document}
	% Title
	\begin{frame}
		\titlepage
	\end{frame}
	% Outline
	\begin{frame}{Outline}
		\tableofcontents
	\end{frame}
	% Section 1
	\section{Introduction to Differential Equations}
		\begin{frame}{Definition of a Differential Equation}
			A differential equation is a mathematical equation
			that relates one or more functions and their derivatives.
			In other words, it describes the relationship between a function
			and its rate of change. The general form of a differential equation is:
				\begin{equation}
					F(x,y,y',y'', \dots, y^{(n)})
				\end{equation}
			Here:
				\begin{itemize}
					\item $x$ is the independent variable.
					\item $y$ is the dependent variable.
					\item $y'$ is the first derivative of $y$ with respect to $x$.
					\item $y''$ is the second derivative, and so on, up to the $n$-th derivative.
				\end{itemize}
		\end{frame}
		\begin{frame}{Types of Differential Equations}
			\begin{itemize}
				\item \textbf{Ordinary Differential Equations (ODEs):} Ordinary Differential Equations (ODEs) can be
					classified into various types based on their properties, structure, and characteristics.
				\item \textbf{Partial Differential Equations (PDEs):} Partial Differential Equations (PDEs) are a type of differential equation that involves multiple
					independent variables and their partial derivatives with respect to those variables.
			\end{itemize}
		\end{frame}
	% Section 2
	\section{First-Order Ordinary Differential Equations (ODEs)}
        \begin{frame}{Separable Equations}
            A separable ordinary differential equation (ODE) is a specific type 
            of differential equation where the variables and their derivatives 
            can be "separated" on each side of the equation.
            This type of equation can be expressed as:
            	\begin{equation}
            		\dfrac{dy}{dx} = g(x)h(y)
            	\end{equation}
            Here, $y$ is the unknown function of $x$, $\frac{dy}{dx}$ is its derivate with respect to $x$, 
            and $g(x)$ and $h(y)$ are known functions.
        \end{frame}
		\begin{frame}{Linear Equations}
			A linear ordinary differential equation (ODE) is an ODE where the dependent variable and 
			its derivatives appear as linear terms.
			The general form of a linear first-order ODE is:
				\begin{equation}
					\dfrac{dy}{dx} + P(x)y = Q(x)
				\end{equation}
			Where $y$ is the dependent variable, $\frac{dy}{dx}$ is its first derivate with respect to $x$, 
			$P(x)$ and $Q(x)$ are known functions of $x$.
		\end{frame}
    % References and bibliography
    \begin{frame}[t,allowframebreaks]{Bibliography}
		 \bibliography{main.bib}
		 	\bibliographystyle{apalike}
		 		\nocite{zill-2012}
	\end{frame}
	% Title
	\begin{frame}[t]
		\maketitle
	\end{frame}
\end{document}